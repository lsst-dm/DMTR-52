\documentclass[DM,lsstdraft,STR,toc]{lsstdoc}
\input meta.tex

\begin{document}

\def\milestoneName{WISE Data Loaded in PDAC}
\def\milestoneId{LDM-503-1}
\def\product{LSST Science Platform}

\setDocCompact{true}

\title[\milestoneId{}~Test Report]{\milestoneId{} (\milestoneName{})~Test Report}
\setDocRef{\lsstDocType-\lsstDocNum}
\setDocDate{\vcsdate}
\setDocUpstreamLocation{\url{https://github.com/lsst/lsst-texmf/examples}}
\author{% `git log --pretty=%an | sort --key=2 | uniq` ?
  Gregory P. Dubois-Felsmann, Xiuqin Wu
}

% Most recent last
\setDocChangeRecord{
\addtohist{}{2018-06-12}{Draft of complete report}{G.~Dubois-Felsmann}
}

\setDocCurator{Gregory P. Dubois-Felsmann}
\setDocUpstreamLocation{\url{https://github.com/lsst-dm/\lsstDocType-\lsstDocNum}}
\setDocUpstreamVersion{\vcsrevision}


\setDocAbstract{
This is the test report for \milestoneId{} (\milestoneName{}), an LSST DM level 2 milestone pertaining to the \product{}, with tests performed according to LSP-00, Portal and API Aspect Deployment of a Wide-Area Dataset.
}

\maketitle

\section{Introduction}
\label{sect:intro}

\subsection{Objectives}
\label{sect:objectives}

The \milestoneId{} milestone calls for the principal WISE primary mission and NEOWISE first-year catalogs to be loaded into the LSP Integration Environment, also known as the Prototype Data Access Center.
The LSP-00 test specification, defined in LDM-540, was used to verify that the data were loaded and that the data service provided meets a subset of requirements for functionality of the API Aspect and the Portal Aspect of the Science Platform.

\subsection{Scope}
\label{sect:scope}

The LSP-00 test specification describes a comprehensive set of tests exploring the functionality of the Science Platform as applied to catalog and image data.
The following test cases were carried out as part of the verification of this milestone:

\begin{description}
\item[LSP-00-00]{Verification of the presence of the expected WISE data}
\item[LSP-00-05]{Demonstration of low-volume and/or indexed queries against the WISE data via API}
\item[LSP-00-10]{Demonstration of table-scan queries against the WISE data via API}
\item[LSP-00-15]{Execution of basic catalog queries in the Portal}
\item[LSP-00-20]{Operation of the UI for interaction with tabular data results}
\item[LSP-00-35]{Linkage of catalog query results to related catalog data}
\end{description}

The LSP-00-25 and LSP-00-30 test cases are not included in this report, as they address image data functionality and the WISE data load to PDAC in the end did not include the image data.
The decision to avoid loading the image data was taken when the time required to load the WISE catalog data turned out to be larger than originally expected, and also required unanticipated hardware upgrades.
In addition to the time required for loading the WISE image data, providing a useful image cutout service would have required developing an LSST-stack (\verb|afw|) camera model for WISE, and this was judged to be an effort with very limited benefit to DM development.
The key usefulness of the WISE data in the DM context is as a scientifically realistic scaling test for Qserv, requiring the handling of tens of billions of rows, so the priorities for the data loading were set accordingly.

The image-oriented test cases can be performed against the SDSS data previously loaded in the PDAC, and will be executed and documented separately.\footnote{As a user convenience, the Portal Aspect in PDAC does provide some limited access to WISE image data directly from IRSA, but as it does not use LSST interfaces to do so, it is not included in this test.}

The WISE catalog data to be loaded included the following tables:

\begin{itemize}

\item AllWISE Source Catalog: this catalog is comparable to the projected LSST \texttt{Object} table, arising from source detection performed on the coaddition of all the data from the pre-hibernation year of WISE operations, including the primary 4-band data as well as shorter periods of 3-band (W1,W2,W3) and 2-band (W1,W2) data taken after the exhaustion of the stored cryogen.

\item AllWISE Multi-Epoch Photometry (MEP) Table: roughly comparable to LSST \texttt{ForcedSource}, arising from forced photometry on every single-epoch image, based on the sources in the AllWISE Source Catalog.
Differs from \texttt{ForcedSource} in that WISE takes all four band images at once, so every row in this table contains data from all four bands.

\item Single-Epoch Photometry tables from four eras of the mission, comparable to LSST \texttt{Source}.
Similarly to the MEP Table, the WISE imaging model produces measurements in each available band in each row.
As the W4 and then W3 bands were shut down as the spacecraft warmed up, their columns were dropped from the table, leading to an evolution of the schema with time.
The four eras and the associated official table names are:
  \begin{itemize}
  \item All-Sky Single Exposure (L1b) Source Table, from the primary cold mission, with all four bands operating;
  \item 3-Band Cryo Single Exposure (L1b) Source Table, from the brief period when the spacecraft was still cold enough to collect W3, but not W4, data;
  \item Post-Cryo Single Exposure (L1b) Source Table, containing only W1 and W2 data, from the period following the primary mission when the cryogen was fully exhausted but operations were continued, after which the spacecraft was put into hibernation until 2013;
  \item NEOWISE-R Year 1 Single Exposure Source Table, from the first year of operations after the reactivation of the spacecraft, and containing only W1 and W2 data.
  \end{itemize}
\end{itemize}

The LSP-00 tests verified that these tables were present on the PDAC and exercised a variety of operations against them in the Portal and API Aspects.
The rest of this report covers the six test cases into which this work was divided.

\subsection{System Overview}
\label{sect:systemoverview}

The \product{} is the part of the LSST Data Management System which presents the LSST Data Products,
a user computational environment, and a set of tools for interacting with them,
to LSST's science users.

The system under test is the initial deployment of the \product{} in the Prototype Data Access Center (PDAC),
an initial application of the NCSA-provided ``integration environment''.
In the tested version of the PDAC,
elements of the database systems (including a Qserv prototype),
a partial deployment of the API Aspect data access Web services (DAX),
and an early version of the Science User Interface and Tools (SUIT) Portal were operated on a hardware cluster managed by NCSA.
The Notebook Aspect was not included in this deployment at all.

Software deployment was done essentially manually on the systems provided by NCSA,
in some cases by traditional software installation and in some using Docker containers,
but no Kubernetes layer was involved.

\subsection{Applicable Documents}
\label{sect:appdocs}

\begin{tabular}[htb]{l l}
% \citeds{LDM-294} & LSST DM Project Management Plan\\
% \citeds{LDM-503} & DM Test Plan\\
% \citeds{LDM-540} & \product{} Test Specification\\
LDM-294 & LSST DM Project Management Plan\\
LDM-503 & DM Test Plan\\
LDM-540 & \product{} Test Specification\\
\end{tabular}


\section{Test Configuration}

\subsection{Documents}

The v1.0 version of LDM-540 was used to guide the tests.
In the process of performing the tests some editing errors were found in LDM-540 and corrections will be submitted.
These mainly affected LSP-00-35, where, due to an editing error,
the test was described as being on the SDSS Stripe 82 data, not the WISE data.

\subsection{Hardware}

The tests were carried out on the initial PDAC hardware configuration at NCSA with its late 2017 upgrade to a more capable server with solid state disk for the Qserv czar node.

\subsection{Software}

A variety of versions of Qserv and DAX software were used in the API Aspect as issues were discovered and corrected.
The final versions, on which all of the tests reported here were repeated, were:

\begin{table}[h]
\centering
\begin{tabular}{l l}
Package & Tag \\ \hline
antlr              &   2.7.7.lsst1  \\
apr                &   1.5.2       \\
apr\_util           &   1.5.4       \\
base               &   14.0-8-g7f6dd6b+6  \\
boost              &   1.66.0      \\
db                 &   14.0+14     \\
doxygen            &   1.8.13.lsst1  \\
flake8             &   3.5.0+2     \\
flask              &   0.12+1      \\
jemalloc           &   5.0.1.lsst1  \\
libcurl            &   7.53.1      \\
libevent           &   2.1.8       \\
log                &   14.0-2-ga5af9b6+11  \\
log4cxx            &   0.10.0.lsst7  \\
lua                &   5.1.4.lsst1-2-gda519a9  \\
mariadb            &   10.1.21.lsst2  \\
mariadbclient      &   10.1.21.lsst2  \\
mysqlproxy         &   0.8.5.lsst1  \\
numpy              &   1.13.1      \\
partition          &   12.1-1-g721d0e5+41  \\
pep8\_naming        &   0.4.1+1     \\
pex\_exceptions     &   14.0-1-g13ef843+10  \\
protobuf           &   2.6.1.lsst4-1-gdbc86f2+1  \\
pybind11           &   2.1.1       \\
pycodestyle        &   2.3.1+1     \\
pyflakes           &   1.6.0       \\
pytest             &   3.2.0.lsst2  \\
pytest\_flake8      &   0.9.1+2     \\
pytest\_forked      &   0.2.lsst2   \\
pytest\_session2file &  0.1.9+3     \\
pytest\_xdist       &   1.20.1.lsst2  \\
python             &   0.0.7       \\
python\_execnet     &   1.4.1.lsst2  \\
python\_future      &   0.16.0+1    \\
python\_mccabe      &   0.6.1+3     \\
python\_mysqlclient &   1.3.7.lsst1+10  \\
python\_psutil      &   5.4.3       \\
qserv              &   tickets.DM-11743-g57c820d962  \\
requests           &   2.9.1.lsst1+2  \\
scisql             &   0.3.8.lsst1+2  \\
scons              &   3.0.0.lsst1+1  \\
sconsUtils         &   14.0-15-g79b7e05  \\
sphgeom            &   12.1-4-g110c6f4+22  \\
sqlalchemy         &   1.0.8.lsst3+4  \\
utils              &   14.0-9-g11010eb+1  \\
xrootd             &   master-g344ca073bb  \\
\end{tabular}
\caption{Database and DAX package versions, last deployed}
\label{tab:dax-versions}
\end{table}


The Portal Aspect used the following software versions, installed in PDAC on January 24, 2018 and retained throughout the testing:

\begin{itemize}
\item{The \verb|firefly| back end package: SHA be6cc56; and}
\item{the \verb|suit| LSST-specific configuration package: SHA 4298ba0.}
\end{itemize}


\section{Personnel}
\label{sect:personnel}

The majority of the test cases were executed by Gregory Dubois-Felsmann (Caltech/IPAC).
All of LSP-00-20 and portions of LSP-00-15 and LSP-00-35 were also executed by Xiuqin Wu (Caltech/IPAC),
from whose work most of the screen shots used arise.


\subsection{Summary Table}
\label{sect:summarytable}

\begin{longtable}
{|p{0.2\textwidth}|p{0.3\textwidth}|p{0.35\textwidth}|}\hline
{\bf TEST CASE ID} & {\bf PASS/FAIL} & {\bf COMMENTS} \\\hline
LSP-00-00 & Pass & Refer to \S\ref{sect:detail-lsp-00-00}. \\\hline
LSP-00-05 & Pass & Refer to \S\ref{sect:detail-lsp-00-05}. \\\hline
LSP-00-10 & Pass & Refer to \S\ref{sect:detail-lsp-00-10}. \\\hline
LSP-00-15 & Pass & Refer to \S\ref{sect:detail-lsp-00-15}. \\\hline
LSP-00-20 & Pass with reservations & Refer to \S\ref{sect:detail-lsp-00-20}. \\\hline
LSP-00-35 & Pass & Refer to \S\ref{sect:detail-lsp-00-35}. \\\hline
\end{longtable}

\subsection{Overall Assessment}
\label{sect:overallassessment}

The test cases executed passed with a variety of minor issues relating to the details of the data import and features not fully implemented in this early version of the Science Platform,
as well as to some rather arbitrary numeric limits in the deployed portal.
The latter affected primarily LSP-00-20 and are described in more detail there.
Overall the test showed that it was possible to import a several $\times 10^8$ row dataset and access it usefully through Qserv and the Portal and API Aspects.

\subsection{Impact of Test Environment}
\label{sect:impact}

As noted above, the deployment of the database, API Aspect, and Portal Aspect components did not use Kubernetes and is not reflective of the planned final system.
This affected not only the management of the components but substantive technical aspects of their behavior;
in particular, in the Kubernetes deployment there will be ingress controllers and other components that affect the connections between components.

Furthermore, the external access control to the components was through a VPN rather than by means of LSP-level authentication and authorization,
and therefore most likely not reflective of parameters such the latency and throughput to be expected from the final system.

No tests of authorization and access control behaviors were performed.


\subsection{Recommended Improvements}
\label{sect:recommendations}

\begin{enumerate}

\item{The current cone-search query size limits in the Portal are unreasonably small and Qserv would clearly support larger ones.
The Portal's polygon search capability already permits the searching of much larger regions with good performance,
so the limitation seems unnecessary.}
\item{The as-deployed Portal limits 2D scatterplots to a maximum of 5,000 points before switching to 2D histograms.
This was motivated by some concerns about earlier versions of the underlying \verb|Plotly| library slowing down visibly for denser scatterplots.
This fallback behavior is not clearly documented or signaled by the UI behavior,
so the user will have some difficulty anticipating it and working around it.
In IRSA production this limit has recently been raised somewhat, while still preserving reasonable performance for larger scatterplots.
This change should be deployed immediately in LSST production.
In addition, a recently developed alternate Web visualization implementation for large scatterplots, based on WebGL, has been developed for \verb|Plotly|.
This implementation is said to scale to over a million points.
If it can be used in Firefly without other significant impacts, we strongly encourage that.
Note that the Firefly tabular data model supports multi-million row tables already.}
\item{The UI functionality related to 2D histograms, called ``heat maps'', should be improved to support more quantitative exploitation.
In particular, the UI should permit the upper and lower bin limits to be specified by the user, not just the number of bins.
If this worked just as it does for 1D histograms, that would be fine.
This would help the user in obtaining simpler bin boundaries.
Also, both types of histograms should provide support for writing out the histogram contents as a table.}
\item{The UI procedure for filtering on manual row selections is not intuitive and requires some attentive exploration in order to discover.  This may need to be rethought.}
\item{The interaction of column filters with row selections needs to be rethought; in particular, row selections should not be cleared by the application of column filters.}
\item{The UI for applying rectangular selections to 2D plots could be improved in two ways:
display moving guide marks along the X and Y axes as the selection is drawn,
facilitating by-eye measurement of the quantitative region boundaries; and once a drawn-rectangle filter has been applied,
its specifications should be accessible in the filter dialogs to allow recording the exact values used.}
\item{The PDAC TAP service does not support standard ADQL and required the use of Qserv-specific functions in order to exploit its performance successfully.
We look forward to the implementation of at least a basic subset of ADQL's geometry constructs.}
\item{The PDAC TAP service should support the VOTable data format.
The service should be tested using publically available validators.}
\item{All tables should be readily searchable by their primary key with good performance.
In the case of the WISE tables there is the oddity that the primary keys appear in both integer and string forms.
For public usability of the WISE service we recommend ensuring that both forms are indexed as needed for good performance.
The data model and database teams should remain alert to the possibility of a similar situation arising in the LSST data model.}
\item{We strongly recommend that there be both programmatic and visual interfaces that aid users in determining whether their queries have triggered, or are likely to trigger, table scans.
This information should also be available from user-oriented documentation about the materialization of release data in the LSP.}

\end{enumerate}

\clearpage

\section{LSP-00-00: Verification of the presence of the expected WISE data}
\label{sect:detail-lsp-00-00}

\input lsp-00-00/lsp-00-00.tex

\clearpage

\section{LSP-00-05: Demonstration of low-volume and/or indexed queries against the WISE data via API}
\label{sect:detail-lsp-00-05}

\input lsp-00-05/lsp-00-05.tex

\clearpage

\section{LSP-00-10: Demonstration of table-scan queries against the WISE data via API}
\label{sect:detail-lsp-00-10}

\input lsp-00-10/lsp-00-10.tex

\clearpage

\section{LSP-00-15: Execution of basic catalog queries in the Portal}
\label{sect:detail-lsp-00-15}

\input lsp-00-15/lsp-00-15.tex

\clearpage

\section{LSP-00-20: Operation of the UI for interaction with tabular data results}
\label{sect:detail-lsp-00-20}

\input lsp-00-20/lsp-00-20.tex

\clearpage

\section{LSP-00-35: Linkage of catalog query results to related catalog data}
\label{sect:detail-lsp-00-35}

\input lsp-00-35/lsp-00-35.tex

\end{document}
